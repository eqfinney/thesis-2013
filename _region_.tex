\message{ !name(THESIS_LATEX_WHAT_THE_HECK.tex)}\documentclass[12 pt]{article}
\usepackage{amsmath,amstext,amsgen,amsbsy,amsopn,amsfonts,graphicx}
\renewcommand{\baselinestretch}{1.5}
\begin{document}

\message{ !name(THESIS_LATEX_WHAT_THE_HECK.tex) !offset(19) }
\section{Introduction}

The importance of galaxy clusters to observational cosmology has long been established.  Historically, because galaxy clusters are the densest objects in the universe's large-scale structure, their observed number density in the universe has been used to constrain certain cosmological models and to more fully understand the development of structure in the universe (Campanelli et al. 2012, Rozo et al. 2010, Sartoris et al. 2012, Hernandez-Monteagudo et al. 1999). In recent years, however, galaxy clusters (and particularly galaxy cluster mergers) have become an especially important laboratory for observational cosmology and particle astrophysics, since these dense and massive structures may yield invaluable information about the interactions between luminous and dark matter in relatively easily observable systems. 

\begin{quote}
\begin{tabular}{|l|l|} \hline
Material & \% Composition \\ \hline
Dark Matter & 78-85\% \\
Intracluster Gas & 11-14\% \\
Galaxies and Other Luminous Matter & 2-6\% \\
\hline
\end{tabular}
\label{tab1}
\end{quote}
Composition of a Galaxy Cluster (Bohringer 2010).

Dissociative galaxy cluster mergers provide especially useful insight into studies of dark matter. In a dissociative merger, when two gravitationally bound mass structures (in this paper, these groups are called “subclusters”) collide, the constituent galaxies and dark matter halos pass directly through the merger, since they are so sparse that a collision between any one of them is extremely unlikely. Intracluster gas, on the other hand, will interact with the surrounding intracluster medium and experience a significant drag force which strips the subcluster of much of its gas (ram pressure stripping), and thus the vast majority of each subcluster\'s gas remains near the center of the merger event, between the post-merger subclusters.

Each of these components of a galaxy cluster (galaxies, dark matter, and intracluster gas) interacts with the other components gravitationally. Thus, when the galaxies and dark matter of one subcluster passes through those of another, it is gravitationally attracted both to the mass of the receding subcluster and to the intracluster gas in between the subclusters. Clearly, the center of the gravitational potential of the system is between the two subclusters. \textbf{I'd still like to add a figure to this effect, even if just something simple.} Then, as evident in Figure \textbf{something (add caption, cite David)}, the subclusters are gravitationally bound to the center of the merger. At a certain point, the subcluster's kinetic energy will be insufficient to continue traveling away from the center of the merger event, and the subcluster will thus begin returning to the center of the gravitational potential. Along the way, friction from intercluster gas and other forces (“dynamical friction”) deplete the system of kinetic energy and cause the subclusters to slow gradually. Even so, the system may oscillate for many millions of years before the kinetic energy is sufficiently dissipated to cause the system to dynamically relax.

This type of merger is ideal for studying the self-interaction of dark matter. If dark matter had a sufficiently large self-interaction cross section, it would interact with itself more strongly than would be expected by gravitational attraction alone; thus, we would expect some sort of offset between self-interacting dark matter and luminous matter (which only interacts with dark matter gravitationally). By determining the offset between the dark matter and luminous matter in a dissociative merger, the strength of the interaction of dark matter with itself may be measured and thus the dark matter self-interaction cross-section may be constrained (Markevitch et al., 2004). \textbf{Cite and add captions to David's other figure.}

Thus, in order to examine the role of dark matter in galaxy cluster mergers, the dynamics of these mergers must be well-understood. Observational studies have traditionally analyzed galaxy cluster merger dynamics by invoking the timing argument (Barrena et al. 2002, Girardi et al. 2008, Bourdin et al. 2011, Boschin et al. 2012), essentially looking at the extreme case in which two clusters begin colliding very far from each other (ie, the gravitational interaction between them is negligible), and are assumed to have a merger axis perpendicular to the line of sight. Several computational studies are more nuanced than this very simplistic timing argument approach, investigating parameters (time since collision, separation distance between subclusters, velocity of each subcluster, etc.) associated with known dissociative mergers. N-body hydrodynamical simulations (Skillman et al. 2013, Vazza et al. 2011) have proven to be useful, but require significant computational time; therefore, this paper invokes a Monte Carlo approach to examining the dynamics of galaxy cluster mergers (Dawson 2012).

Within this approach, further data can be incorporated into an understanding of the attributes of galaxy cluster mergers, and some of this evidence comes in the form of long arcs of radio emission along one or both sides of the merger axis of some dissociative mergers (van Weeren et al. 2010, 2011b,c, 2012). The leading contemporary theories postulate that these arcs are created by synchrotron emission from accelerated electrons during the shock of a collision of subclusters (Ensslin et al. 1998, Roettiger et al. 1999, Venturi et al. 1999). This emission is polarized perpendicular to the direction of the magnetic field along which it occurs, so the degree to which an extended radio source appears polarized can constrain the angle at which the system is viewed. The magnetic fields within the intracluster gas of a merging galaxy cluster are usually quite disorganized, and when this gas is compressed during a shock, the magnetic fields can be aligned perpendicular to the line of sight of the merger axis but remain disorganized along the line of sight. Then if a system's radio relics display a high polarization fraction, it can limit the system to having a viewing angle nearly perpendicular to the line of sight. In this manner, the degree of polarization (which is strongly affected by the disorganization of the magnetic fields) can be used to provide an upper bound on the viewing angle of the magnetic field (Ensslin et al. 1998, van Weeren et al. 2011a, Skillman et al. 2013).

In this work, the the radio relics around the merging galaxy cluster 1 RXS J060313.4+421231 (the “Toothbrush Cluster”) are examined and incorporated into previously established dynamical models of this cluster. The spectral index and polarization values of the radio relics associated with this cluster are used as priors in a Bayesian analysis of the system in order to constrain the viewing angle, collision velocity, time since collision, and other relevant parameters. The paper will be organized as follows: Section 2 examines the historical background of this project (Section 2.1 examines the history of dissociative merger and radio relic studies and Section 2.2 examines the history of methods used to study these structures). The data on the Sausage and Toothbrush clusters and the code used to analyze each of them are described in Section 3. In Section 4, results are presented, and they are discussed and related to major issues in cosmology in Section 5. The paper is concluded with Section 6. 

\message{ !name(THESIS_LATEX_WHAT_THE_HECK.tex) !offset(135) }

\end{document}
